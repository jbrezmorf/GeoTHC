% % ***************************************** SYMBOLS
%  \def\abs#1{\lvert#1\rvert}
%  \def\argdot{{\hspace{0.18em}\cdot\hspace{0.18em}}}
 \def\avg#1{\left\{#1\right\}}
%  \def\D{{\tn D}}
%  \def\div{\operatorname{div}}
 \def\Eh{\mathcal E_h}       % edges of \Th
 \def\Ehb{\mathcal E_{h,B}}  % edges of \Th on boundary
 \def\Ehcom{\mathcal E_{h,C}}         % edges of \Th on interface with lower dimension
 \def\Ehdir{\mathcal E_{h,D}}         % Dirichlet edges of \Th
 \def\Ehint{\mathcal E_{h,I}}       % interior edges of \Th
 \def\Ehneu{\mathcal E_{h,N}}         % Neumann or Robin edges of \Th
%  \def\grad{\nabla}
 \def\jmp#1{[#1]}
 \def\nn{\vc n}
%  \def\vc#1{\mathbf{\boldsymbol{#1}}}     % vector
%  \def\R{\mathbb R}
%  \def\sc#1#2{\left(#1,#2\right)}
%  \def\Th{\mathcal T_h}       % triangulation
%  \def\th{\vartheta}
%  \def\tn#1{{\mathbb{#1}}}    % tensor
%  \def\Tr{\operatorname{Tr}}
 \def\wavg#1{\avg{#1}_\omega}
%  \def\where{\,|\,}
\def\bb{\vc b}
% %***************************************************************************
% 

Modely popsané v sekcích \ref{sc:transport_model} a \ref{sc:heat} lze souhrnně zapsat jako abstraktní systém rovnic advekce-difúze na oblastech $\Omega_d$, $d=1,2,3$, propojených komunikačními členy.
Uvažujme tedy pro $d=1,2,3$ rovnici
\begin{subequations}
\label{eq:abstr_system}
\begin{equation}
\partial_t u_d + \div(\bb u_d) - \div(\tn A\nabla u_d) = f + q(u_{d+1},u_d) \mbox{ v }\Omega_d
\end{equation}
s počátečními a okrajovými podmínkami
\begin{align}
u_d(0,\cdot) &= u^0 &&\mbox{ v }\Omega_d,\\
\label{eq:bc_abstr_dir} u_d &= u^D &&\mbox{ na }\Gamma^D_d,\\
\label{eq:bc_abstr_neu} (\bb u_d-\tn A\nabla u_d)\cdot\nn &= f^N + \sigma^R(u_d - u^D) &&\mbox{ na }\Gamma^N_d,\\
(\bb u_d-\tn A\nabla u_d)\cdot\nn &= q(u_d,u_{d-1}) &&\mbox{ na } \Gamma^C_d:=\overline\Omega_d\cap\overline\Omega_{d-1}.
\end{align}
Komunikační člen $q(u_{d+1},u_d)$ má tvar
\begin{equation}
q(u_{d+1},u_d) = \begin{cases}\alpha u_{d+1} + \beta u_d & \mbox{ v }\Gamma^C_{d+1},~d=1,2,\\ 0 & \mbox{ mimo }\Gamma^C_{d+1}\mbox{ a pro }d=3.\end{cases}
\end{equation}
\end{subequations}
Systém \eqref{eq:abstr_system} je diskretizován v prostoru pomocí nespojité Galerkinovy metody s váženými průměry, která byla odvozena pro případ jedné oblasti v \cite{ern_stephansen_zunino} (pro a posteriorní odhady viz \cite{ern2010guaranteed}).
Pro časovou diskretizaci je použito implicitní Eulerovo schéma.


Nechť $\tau$ značí délku časového kroku a $h$ prostorový diskretizační parametr.
Pro regulární dělení $\Th^d$ oblasti $\Omega^d$, $d=1,2,3$, na simplexy, jehož norma (nejdelší hrana) je $h$, definujeme následující množiny stěn elementů:
\begin{align*}
&\Eh^d &&\mbox{stěny všech elementů v $\Th^d$ (tj. trojúhelníků pro $d=3$, úseček pro $d=2$ a uzlů pro $d=1$)},\\
&\Ehint^d &&\mbox{vnitřní stěny},\\
&\Ehb^d &&\mbox{hraniční stěny},\\
&\Ehdir^d(t) &&\mbox{stěny, kde je předepsána Dirichletova podmínka \eqref{eq:bc_abstr_dir}},\\
&\Ehneu^d(t) &&\mbox{stěny, kde je předepsána Neumannova nebo Robinova podmínka  \eqref{eq:bc_abstr_neu}},\\
&\Ehcom^d &&\mbox{stěny koincidující s $\Gamma^C_d$}.
\end{align*}
Pro vnitřní stěnu $E$ označme symboly $T^-(E)$ a $T^+(E)$ elementy sdílející $E$.
Symbolem $\n$ pak rozumíme jednotkový normálový vektor k $E$ směřující z $T^-(E)$ do $T^+(E)$.
Rozdíl hodnot funkce $f$ mezi sousedícími elementy definujeme jako $\jmp{f}=f_{|T^-(E)}-f_{|T^+(E)}$, podobně průměr $\avg{f}=\frac12(f_{|T^-(E)} + f_{|T^+(E)})$ a vážený průměr $\wavg{f}=\omega f_{|T^-(E)} + (1-\omega) f_{|T^+(E)}$.
Váha $\omega$ je volena specifickým způsobem (viz \cite{ern_stephansen_zunino}) s ohledem na možnou nehomogenitu tenzoru $\tn A$.

% Let us fix one substance and the space dimension $d$.
V každém časovém kroce $t_k=k\tau$ hledáme diskrétní řešení $u_d^{h,k}\in V_d^h$, kde
$$ V_d^h = \{v:\overline{\Omega^d}\to\R\where v_{|T}\in P_1(T)~\forall T\in\Th^d\} $$
je prostor po částech afinních funkcí na elementech $\Th^d$, obecně nespojitých na rozhraních elementů.
Počáteční podmínka pro $u_d^{h,0}$ je nastavena jako $L^2$-projekce funkce $u^0$ na $V_d^h$.
Pro $k=1,2,\ldots$ je $u_d^{h,k}$ dáno jako řešení úlohy
\begin{equation*}
\frac1\tau\sc{u_d^{h,k}-u^{h,k-1}_d}{v}_{\Omega^d} + a^{h,k}_d(u^{h,k}_d,v) = b^{h,k}_d(v) \quad \forall v\in V^h_d.
\end{equation*}
Zde $\sc{f}{g}_{\Omega^d}=\int_{\Omega^d} f g$, a formy $a^{h,k}_d$, $b^{h,k}_d$ jsou definovány následovně:
\begin{align*}
a^{h,k}_d(u,v) = &\sc{\tn A\nabla u}{\nabla v}_{\Omega^d}
- \sc{\bb u}{\nabla v}_{\Omega^d}\\
&- \sum_{E\in\Ehint^d}\bigg(\sc{\wavg{\tn A\nabla u}\cdot\n}{\jmp{v}}_E + \Theta\sc{\wavg{\tn A\nabla v}\cdot\n}{\jmp{u}}_E
+ \sc{\bb\cdot\n\avg{u}}{\jmp{v}}_E
+ \gamma_E\sc{\jmp{u}}{\jmp{v}}_E\bigg)\\
&+ \sum_{E\in\Ehb^d}\sc{\bb\cdot\n u}{v}_E
+ \sum_{E\in\Ehneu^d(t_k)}\sc{\sigma^R u}{v}_E\\
&+ \sum_{E\in\Ehdir^d(t_k)}\bigg(\gamma_E\sc{u}{v}_E - \sc{\tn A\nabla u\cdot\nn}{v}_E - \Theta\sc{\tn A\nabla v\cdot\nn}{u}_E\bigg),\\
% \end{multline*}
% 
% \begin{equation*}
b^{h,k}_d(v) = &\sc{f+q}{v}_{\Omega^d} + \sum_{E\in\Ehdir^d(t_k)}\bigg(\gamma_E\sc{u^D}{v}_E - \Theta\sc{u^D}{\tn A\nabla v\cdot\nn}_E\bigg)\\
&+ \sum_{E\in\Ehneu^d(t_k)}\sc{f^N+\sigma^R u^D}{v}_E
+ \sum_{E\in\Ehcom^d(t_k)}\sc{q}{v}_E.
\end{align*}
Dirichletova podmínka je zde vynucena penalizací, přičemž parametr penalizace $\gamma_E>0$ je volitelný; jeho hodnota ovlivňuje míru nespojitosti řešení. Pro $\gamma_E\to+\infty$ tak (alespoň formálně) asymptoticky dostáváme metodu konečných prvků.
Konstanta $\Theta$ může nabývat hodnot $-1$, $0$ nebo $1$, což odpovídá nesymetrické, neúplné a symetrické variantě nespojité Galerkinovy metody.

% \paragraph{Communication between regions of the same dimension.}
V případě nižších dimenzí ($\Omega^1$, $\Omega^2$) je možné uvažovat komplexnější topologii, kdy jedna stěna je sdílena více než 2 elementy.
Pro tuto situaci je uvedená metoda zobecněna za předpokladu tzv. ideálního mísení.
Nechť stěna $E$ je sdílena elementy $T_1^-,\ldots,T_{n_i}^-,T_1^+,\ldots,T_{n_o}^+$, kde $T_i^-$ značí vtokové elementy ($\bb\cdot\nn\ge0$) a $T_i^+$ výtokové elementy ($\bb\cdot\nn<0$).
Označme $q_i:=(\bb\cdot\nn)_{|T_i}$ výtok z $T_i$ a definujme $I^-:=\{i\where q_i\le 0\}$ a $I^+:=\{i\where q_i>0\}$ množiny indexů všech výtokových, resp. vtokových elementů.

Pro každou dvojici $(i,j)\in I^+\times I^-$ pak definujeme tok z $T_i$ do $T_j$ jako
$$ q_{i\to j} := \frac{q_i q_j}{\sum_{k\in I^-}{q_k}}. $$
V bilineární formě $a_d^{h,k}$ pak výraz $\sc{\bb\cdot\nn\avg{u}}{\jmp{v}}_E$ nahradíme výrazem
$$ \sum_{(i,j)\in I^+\times I^-}\sc{q_{i\to j}\avg{u}}{\jmp{v}}_E, $$
kde operátory $\avg{\cdot}$ a $\jmp{\cdot}$ jsou uvažovány vzhledem ke dvojici elementů $(T_i,T_j)$.

